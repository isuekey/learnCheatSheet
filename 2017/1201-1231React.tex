\documentclass{book}
\usepackage{CJK}
\begin{document}
\begin{CJK*}{UTF8}{gbsn}
\title{React Basic Reference}
\author{刘翰儒}
\date{2017-11-30}
\maketitle
\tableofcontents
\chapter{Quick Start}
\section{Installation} React是弹性的,能用在不同的项目中。可以用它来创建新的APP,也可以在现有的代码库中逐渐引入,还不用重写现有的全部代码(部分修改是必须的)。
\paragraph{有几种开始原因}
\subparagraph{体验一下}
\subparagraph{放下成见,创建一个新的APP}
\subparagraph{现有APP中增加React}
\subsection{Try Out React}
\paragraph{}如果只是感兴趣,可以试试CodePen。在这里你不需要安装任何东西,只需要写代码,看看他是怎么工作的。
\paragraph{}如果习惯用文本编辑器,可以下载这个HTML文件,编辑,然后重新在浏览器中打开。这样做的表现很慢,不要在产品中这样使。
\paragraph{}如果你是为了一个完整的应用,通常有两种方式来开始:使用Create React App或者加入到心有应用中。
\subsection{创建一个新应用}
\paragraph{CreateReactApp}是创建新的单页面React应用的最佳方式。他会配置开发环境,以便于使用最新的JavaScript特性,好的开发体验和优化成品应用。同时需要Node$ >=6 $
\begin{verbatim}
npm install -g create-react-app
create-react-app my-app
cd my-app
npm start
\end{verbatim}
\paragraph{如果npm是5.2.0+,可以使用npx}
\begin{verbatim}
npx create-react-app my-app
cd my-app
npm start
\end{verbatim}
\paragraph{Create React App}不涉及后段逻辑和数据库。只是提供了前端的构建路线,所以可以配合任意后代。构建的时候需要用到类似Babel和Webpack这样的工具,但是工作的时候不需要任何设置。
\paragraph{}但需要部署到生产环境的时候,执行$ npm run build $会在build目录创建一个优化过的应用。可以通过其README或者User Guide获得更多屏住。
\subsection{在现有应用中添加React}
\paragraph{增加React无需重写应用}
\paragraph{尽量从某个小的部分开始,比如独立的插件,看看是否满足需要}
\paragraph{尽管React可以不用需要构建路线,还是建议使用这些工具,让自己产品更加专业。现代构建路线包括:}
\subparagraph{包管理工具:npm或者yarn}更方便管理第三方包。
\subparagraph{捆绑机:webpack或者Browserify等}把模块化代码捆绑成小的包,优化加载时间。
\subparagraph{编译器:Babel}将现代ES6等的代码编译成ES5的代码,让他可以在旧的浏览器中运行。
\subsubsection{安装React}
\paragprah{Note}强烈推荐设置一个产品构建配置,确保在用快速的React产品。
\subsubsection{使用ES6与JSX}
\paragraph{}推荐配合Babel使用React,这可以在代码中使用ES6和JSX。ES6的现代JavaScript特性让开发更容易,而JSX让JavaScript更容易用React工作。
\paragraph{}Babel安装介绍说明了在不同环境中如何设置。在React项目中确认安装了babel-preset-react,babel-preset-env,并且在.babelrc中应用了这些工具。
\subsubsection{用ES6和JSX赞叹世界}
\paragraph{}推荐使用webpack或者Broserify等捆绑器,这就可以让模块化的代码捆绑成一个小的包裹中,优化加载时间。
\paragraph{}代码如下:
\begin{verbatim}
import React from 'react';
import ReactDom from 'react-dom';
ReactDom.render(
<h1> Hello, World! </h1>,
document.getElementById('root')
);
\end{verbatim}
\paragraph{} 上述代码会渲染到带有id为root的DOM元素中,所以需要在HTML文件中加入\verb | <div id=''root''></div> |
\paragraph{}类似的,React Component也可以渲染到某个DOM元素中。
\paragraph{}学习更多的整合React到现有代码中
\subsubsection{开发与产品版本}
\paragraph{}React默认带有很多警告,这些警告在开发中很有用。
\paragraph{}然而这会让开发版远远大于产品版的应用。
\paragraph{}学习如何告诉网页使用正确的React,如何最有效的配置产品构建过程
\subparagraph{}用Create React App创建产品构建
\subparagraph{}用Single-File构建器创建产品构建
\subparagraph{}用Brunch创建产品构建
\subparagraph{}用Browserify创建产品构建
\subparagraph{}用Rollup创建产品构建
\subparagraph{}用webpack创建产品构建
\subsubsection{使用CDN}
\paragraph{}如果不想用npm来管理包,react和react-dom都提供了单一的umd文件(CDN支持)
\begin{verbatim}
<script crossorigin src="https://unpkg.com/react@16/umd/react.development.js"></script>
<script crossorigin src="https://unpkg.com/react@16/umd/react-dom.development.js"></script>
\end{verbatim}
\paragraph{}正像文件名中所诉,这是为了开发准备,不适用与产品。微型化、优化后的React版本,应如下引入
\begin{verbatim}
<script crossorigin src="https://unpkg.com/react@16/umd/react.production.js"></script>
<script crossorigin src="https://unpkg.com/react@16/umd/react-dom.production.js"></script>
\end{verbatim}
\paragraph{}如果要特定版本react和react-dom,替换上述脚本中的16为指定版本数字即可。
\paragraph{}如果用Bower,可以通过react包来管理。
\paragraph{}为啥要用crossorigin
\paragraph{}如果要从CDN上用React,推荐保留crossorigin描述。
\begin{verbatim}
<script crossorigin src="..."></script>
\end{verbatim}
\paragraph{}推荐确认CDN中返回头是否带有Access-Control-Allow-Origin:*。
\paragraph{}这么做在React16里会有更好的错误捕获体验。
\section{Introducing JSX}
\subsection{Lists and Keys}
\subsection{Rendering Elements}
\subsection{Components and Props}
\subsection{State and Lifecycle}
\subsection{Composition vs Inheritance}
\subsection{Conditional Rendering}
\subsection{Forms}
\subsection{Handling Events}
\subsection{Lifting State Up}
\subsection{Tinking in React}

\chapter{Reference:Component}
\subsubsection{componentDidCatch()}
\subsubsection{componentDidMount()}
\subsubsection{componentDidUpdate()}
\subsubsection{compoenntWillMount()}
\subsubsection{componentWillReceiveProps()}
\subsubsection{compoenntWillUnmount()}
\subsubsection{componentWillUpdat()}
\subsubsection{constructor()}
\subsubsection{defaultProps}
\subsubsection{displayName}
\subsubsection{forceUpdate()}
\subsubsection{props}
\subsubsection{render()}
\subsubsection{setState()}
\subsubsection{shouldComponentUpdate()}
\subsubsection{state}

\chapter{Reference:React}
\subsubsection{cloneElement()}
\subsubsection{createElement()}
\subsubsection{createFactory()}
\subsubsection{isValidElement()}
\subsubsection{React.Children}
\subsubsection{React.Children.count}
\subsubsection{React.Children.map}
\subsubsection{React.Children.only}
\subsubsection{React.Children.toArray}
\subsubsection{React.Component}
\subsubsection{React.PureComponent}

\chapter{Reference}
\subsubsection{createPortal()}
\subsubsection{DOM Element}
\subsubsection{findAllInRenderedTree()}
\subsubsection{findDOMNode()}
\subsubsection{findeRenderedComponentWithType()}
\subsubsection{findRenderedDOMComponentWithClass()}
\subsubsection{findeRenderedDOMComponentWithTag()}
\subsubsection{Glossary of React Terms}
\subsubsection{hydrate()}
\subsubsection{isCompositeComponent()}
\subsubsection{isCompositeComponentWithType()}
\subsubsection{isDOMComponent()}
\subsubsection{isElement()}
\subsubsection{JavaScript Environment Requirements}
\subsubsection{mockComponent()}
\subsubsection{props}
\subsubsection{props.children}
\subsubsection{React Top-Level API}
\subsubsection{React.Component}
\subsubsection{ReactDOM}
\subsubsection{ReactDOMServer}
\subsubsection{render()}
\subsubsection{renderIntoDocument()}
\subsubsection{renderToNodeStream()}
\subsubsection{renderToStaticMarkup()}
\subsubsection{renderToStaticNodeStream()}
\subsubsection{renderToString()}
\subsubsection{scryRenderedComponentsWithType()}
\subsubsection{scryRenderedDOMComponentsWithClass()}
\subsubsection{scryRenderedDOMComponentsWithTag()}
\subsubsection{Shallow Renderer}
\subsubsection{shallowRenderer.getRenderOutput()}
\subsubsection{shallowRenderer.render()}
\subsubsection{Simulate}
\subsubsection{state}
\subsubsection{SyntheticEvent}
\subsubsection{Test Renderer}
\subsubsection{Test Utilities}
\subsubsection{testInstance.children}
\subsubsection{testInstance.find()}
\subsubsection{testInstance.findAll()}
\subsubsection{testInstance.findAllByProps()}
\subsubsection{testInstance.findByProps()}
\subsubsection{testInstance.findByType()}
\subsubsection{testInstance.instance()}
\subsubsection{testInstance.parent}
\subsubsection{testInstance.props}
\subsubsection{testInstance.type}
\subsubsection{TestRenderer.create()}
\subsubsection{testRenderer.getInstance()}
\subsubsection{testRendererroot}
\subsubsection{testRenderer.toJSON()}
\subsubsection{testRenderertoTree()}
\subsubsection{testTrenderer.unmount()}
\subsubsection{testRenderer.update()}
\subsubsection{unmountcomponentAtNode()}

\chapter{Advanced Guides}
\subsubsection{Accessibility}
\subsubsection{Context}
\subsubsection{Error Boundaries}
\subsubsection{Higher-Order Components}
\subsubsection{Intergrating with Other Libraries}
\subsubsection{JSX In Depth}
\subsubsection{Optimizing Performance}
\subsubsection{Portals}
\subsubsection{React Without ES6}
\subsubsection{React Without JSX}
\subsubsection{Reconciliation}
\subsubsection{Refs and the DOM}
\subsubsection{Static Type Checking}
\subsubsection{TypeChecking With PropTypes}
\subsubsection{Uncontrolled Components}
\subsubsection{Web Components}
\end{CJK*}
\end{document}
